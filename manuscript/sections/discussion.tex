\section{Discussion}
\label{sec:discussion}


\subsection{Timing effects}

What if not only the probability to find a vulnerability decreases, but it takes an increasing amount of time? 



Limited information on time between vulnerability submission and 


\subsection{public versus private programs}


\subsection{Renewal as code changes}



{\bf Is it bad or good if a researcher submits a duplicate bug?  It's probably bad, but still the learning component remains ? (in relation to all-pay auctions?)}



{\bf Correlation between reputation and capacity to discover bugs with high rank ?}

{\bf Cascades, learning, re-inforcing process}

{\bf our results suggest that program managers should enroll a lot of security researchers early on, if not before the start of the program, to maximize the initial pool of researchers who in turn can compound best on their cumulative knowledge i.e., $t^{1.4 - 1.27}$}

{\bf now bounty program managers face a tradeoff: the more the initial load, the longer the response time in order to go through all submissions (show evidence)}

{\bf strange thing: on the contrary of usual crowdsourcing, here the security researcher has access to the same information as the user of the software platform.}


{\bf it remains to be explained the distribution of number of vulns discovered per participant: critical cascades?}



\cite{benkler2011penguin}

We present a model of workers supplying labor to paid crowdsourcing projects. We also introduce a novel method for estimating a worker's reservation wage - the key parameter in our labor supply model. We tested our model by presenting experimental subjects with real-effort work scenarios that varied in the offered payment and difficulty. As predicted, subjects worked less when the pay was lower. However, they did not work less when the task was more time-consuming. Interestingly, at least some subjects appear to be "target earners," contrary to the assumptions of the rational model. The strongest evidence for target earning is an observed preference for earning total amounts evenly divisible by 5, presumably because these amounts make good targets. Despite its predictive failures, we calibrate our model with data pooled from both experiments. We find that the reservation wages of our sample are approximately log normally distributed, with a median wage of \$1.38/hour. We discuss how to use our calibrated model in applications. \cite{horton2010labor}

