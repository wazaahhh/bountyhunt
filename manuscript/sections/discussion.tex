\section{Discussion}
\label{sec:discussion}
Finding bugs in software code is one of the oldest and toughest problems in computer science. While algorithm based approaches have been developed over the years, human verification has remained a prime approach to debugging and vulnerability hunting. Moreover, the idea of getting ``enough eyeballs" to inspect the code has been a cornerstone argument for the open source software movement \cite{raymond1999cathedral} along with the full-disclosure argument. Bug bounty programs are perhaps the latest successful incarnation of markets for trading bugs and vulnerabilities \cite{bohme2006comparison}, as specific form of reverse Dutch book auction \cite{ozment2004bug}, which sets incentives to disclose early through a form of monopsony, combined with an increase of payoff for more rare and difficult bugs.\\

Here, we have found that the number of bug and vulnerabilities is super-linearly associated with the number of security researchers. However, the distribution of bugs found per researcher is highly skewed, hence suggesting that while many researchers find only few bugs, a hand of bug hunters do particularly well. This result is reminiscent of productive bursts in open source software development \cite{sornette2014much}. Skewed performance may be attributed to critical cascade mechanisms, but its origins remain unknown. Performance within a bug bounty program most likely combines some {\it ex-ante} fitness, endogenous learning capabilities, and perhaps some pure luck in the bug search process, which all need further detail investigation and disentanglement.\\

Most security researchers however participate in multiple bug bounty programs, and when a new program is launched, they face the strategic choice of switching program. Our results show that researchers have a decreasing incentive to explore higher ranks within the same program (Figure \ref{fig:scalings_awards}B), while they have an increasing, yet marginally decreasing, incentive to explore multiple programs (Figure \ref{fig:scalings_awards}C). We further confirm that researchers switch when new programs are launched. This is a strong signal that researchers make rational choices in the bug hunting environment: They have high incentives to switch quickly to a new program and harvest as fast as possible many frequent bugs with little reward, rather than less frequent yet more endowed vulnerabilities, even though the reward structure of bug bounty programs seems to incentivize generously the reward of high rank (i.e., less probable) vulnerabilities (Figure \ref{fig:scalings_awards}A).\\

This result raises questions on some hard limits associated with incentives associated with bug discovery by humans: The disincentive clearly stems from the difficulty for individual researchers to reach high ranks (i.e., $P_{k}$), not from the reward scheme. However, while it may not be a good deal on average for researchers (i.e., following the expected payoff) to pursue exploration of the same program, some specially skilled or lucky researchers have a chance to strive towards high rank high yield bounties. Hence, increasing the enrollment mechanically increases the chance to find those special researchers from a larger pool of researchers.\\

Using {\it rankings} provides handy insights on the processes governing the vulnerability discovery process, and to some extent, associated incentives. However, the rank is an arbitrary measure of time, which hardly accounts for the effort spent on researching bugs, as well as for discounting effects. For instance, if the time required to find a vulnerability increases with the rank, then the expected payoff shall be discounted accordingly. Other aspects enter the equation: While most submissions occurs early on after the program launch, this is also the moment when an organization might be less prepared to respond to a large flow of tasks, which in turn may trigger priority queueing and contingent delays \cite{maillart2011quantification}. While some workaround may be envisioned, publicly available data currently limit some desirable investigations, involving timing and discounting effects.\\

In this study, we have considered the incentive mechanisms at the aggregate level. Managers however organize enrollment, set incentives and tackle the operational pipeline, involving submission reviews and payroll processing. All these aspects unique to each program may crowd in, or on the contrary crowd out, security researchers, in particular their willingness to participate to start with, but also the amount of effort they are ready to throw in the search of vulnerabilities. It is the hope of the authors to get increasingly fine-grained insights and establish benchmarks of most performing bug bounty programs, in an environment driven by large deviation statistics, with outlier contributions, and incentives structures, which resemble the St-Petersburg paradox, a well-known puzzle for decision making in behavioral economics.
