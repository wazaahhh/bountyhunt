\section{Discussion}
\label{sec:discussion}
Finding bugs in software code is one of the oldest and toughest problems in software engineering \cite{adams1984textordfeminineoptimizing}. While algorithm-based approaches have been developed over the years \cite{avgerinos2014enhancing}, human verification has remained a prime way for bug hunting. While resorting to the crowd for finding bugs is not new \footnote{The first known bug bounty is the reward check program implemented by Donald Knuth in 1985 for debugging his book {\it TeX: The Program} ({\it http://truetex.com/knuthchk.htm}).}, bug bounty programs have recently been promoted by the emergence of bug bounty platforms. Here we have studied the incentive mechanisms across 35 bug bounty programs on HackerOne. Our results show that the number of discovered bugs and vulnerabilities in a bounty program is super-linearly associated with the number of security researchers. However, the distribution of bugs found per researcher per program is bounded: in a given bug bounty program, the marginal probability of finding additional bugs is decreasing rapidly. On the contrary, security researchers have high incentive to switch among multiple bug bounty programs. We find indeed that each newly launched program has a negative effect on submissions to incumbent bug bounty programs. Furthermore, controlling specifically for monetary incentives, we find that the amount of reward for valid bugs in newly launched programs has a negative effect on the number of bug submissions to incumbent programs. These results bring further insights on theory of bug discovery. They also help draw practical organization design recommendations for bug bounty platforms such as HackerOne, as well as for organizations managing bug bounty programs.\\
 
Our results provide an essential validation step to the theory formulated by Brady et al. \cite{brady1999murphy}. No single security researcher is able to find most bugs in one program. On the contrary, a good bug bounty program involves submissions from a diverse crowd of security researchers. Borrowing to the formulation by Brady et al. of software security as a phenomenon of evolutionary pressure dictated by environmental changes, we shall propose that any additional security researcher involved in a bug bounty program brings a unique combination of skills and mindset. This unique perspective is comparable to a slightly changing environment for the software piece under scrutiny, and is associated with unique opportunities for each security researcher, regardless of the opportunity level of other researchers focused on the same software piece. Yet because people cannot easily change their skills and mindset, once the opportunity has been exploited, finding additional bugs gets much harder. Therefore, researchers tend to turn attention to newly launched bug bounty programs.\\

The fact that each security researcher has a limited capacity to uncover a large number of bugs for a specific program (i.e., on a specific piece of software) carries a strong justification for the existence of bug bounty programs as a tool for engaging a large and diverse crowd of security researchers, beyond internalized software testing and security research teams. As a concrete case, we discuss the {\it Uber} bug bounty program launched in 2016 \cite{moussouris2016}. Uber has designed its program as a way to select and hire a band of {\it security czars} from a larger crowd. Although there is certainly nothing wrong with hiring a security expert in an opportunistic manner, our results rather suggest that systematically using bug bounty programs for hiring may prove counter-productive in the long term. Security researchers will be likely to tune their expectations and behaviors toward getting a job. The approach implemented by Uber may reduce engagement by researchers who do not expect to get a job offer. Thus, it could limit the involvement of a larger crowd and its renewal as more permanent security positions get filled. \\

The strategy followed by Uber for its bug bounty program is also interesting from a theoretical perspective: Uber is following a well-known strategy first described by Ronald Coase \cite{coase1937}, which prescribes that if an organization has repeated interactions on the market with the same counterpart -- or {\it ceteris paribus} a similar counterpart -- then the organization is better off internalizing the resource in order to avoid repeated transaction costs. Hence Uber considers security researchers as {\it substitute goods}, while our results -- and actually the mere existence of bug bounty programs -- rather demonstrate that security researchers are {\it complements}. The distinction between substitutes and complements regarding security researchers brings a fundamental justification for the existence and the future development of bug bounty programs as market places for trading bugs and vulnerabilities \cite{bohme2006comparison}. \\

Yet the organization designs of bug bounties programs and the online platforms supporting them are still pretty much empirical. The validation step performed here provides additional theoretical insights. On the one hand, these insights shall be useful to formulate recommendations. On the other hand, they help identify blind spots, which will deserve future theoretical and empirical research efforts. We propose 3 major recommendations that may significantly improve the efficiency of bug bounty programs and the online platforms hosting them, as well as help maximize the engagement of security researchers. 

\subsection{Encourage enrollment, mobility and renewal}
Bug bounty programs shall encourage mobility by devoting resources to the recruitment of security researchers who were not previously involved in the program, rather than increasing efforts to keep security researchers who have already performed well. Mobility increases chances to find security researchers with diverse skills and mindset, who in turn will find additional bugs. Similarly, bug bounty platforms have the possibility to encourage mobility across bug bounty programs. We have found that mobility across programs already exists, in particular mobility from old to newly launched programs. We also advocate the active recruitment of new security researchers by both bug bounty programs and platforms. For example, the bug bounty platform may highlight older programs to security researchers who have recently enrolled on the platform.

\subsection{Feature major changes for front-loading}
The launch of a new bug bounty program is a unique moment, which can attract a large number of security researchers. Yet front-loading may also be organized when a software piece receives a major update with higher probability of finding bugs. For instance, when a new software release contains significant changes in the codebase, bug bounty program managers should feature these changes and help security researchers focus on issues they have not previously been exposed to. This approach may also help bug bounty program managers steer the attention of security researchers toward more pressing security issues. Front-loading can also be organized with a temporary increase of bug bounty rewards. Additionally, dynamic adaptation of incentives can help manage contingencies associated with surges of submissions. For instance, bug bounty managers may decide to reduce rewards during internal overload periods. 
 
\subsection{Organize fluid and low-transaction cost markets}
One overarching concern with trading bugs and vulnerabilities is the temptation by security researchers to consider selling their bugs on the black market. One way to alleviate this problem is to streamline transactions costs associated with bug submission and reward operations. Encouraging mobility without providing market fluidity indeed exposes to the risk that bugs get sold more often on the black market. Our recommendation is reminiscent of the strategy followed by {\it Apple Inc}: by offering cheap enough online music, easy to download on {\it iTunes}, Apple Inc. was able to capture most of the online music black market, such as {\it Napster}. Bug bounty programs face similar challenges and opportunities to capture a larger market share by reducing transaction costs and thus offering an alternative to uncertainties associated with the black market.\\


Our recommendations stem directly from the validation step we performed. They show the importance of having a clear view of the theoretical and empirical underpinnings of the mechanisms of bug bounty programs and of platforms organizing them. Mobility, front-loading and market fluidity may be organized either through top-down bureaucracy or by setting market incentives appropriately. The relative advantages of bureaucracy and market organization shall be further studied. Our recommendations for designing bug bounty programs apply indiscriminately to public and private bug bounties. However, we believe that private bug bounty programs face more complex challenges as they select their invited participants. The selection process is costly and {\it de facto} reduces the pool of security researchers.\\

%We also suggests that both individual bug bounty programs and bug bounty platforms shall aim at constantly attracting more researchers, in light of the results in Section~\ref{sec:enrollment}. Previous work~\cite{zhao2015empirical} has made several suggestions, such as a first time bonus, to achieve this goal. We further suggest that current private bounty programs should gradually increase their enrollment and eventually go public. Receiving no bugs in the private stage does not indicate that the system is secure. Rather, it is more likely that the system has not been fully tested by the crowd. In addition, bug bounty platforms should improve their researcher invitation/allocation mechanisms to encourage the flow of diverse researchers to different programs~\cite{zhao2016crowdsourced}.


Our study would benefit from additional empirical research using data which are currently not available from public sources. First, our results are limited by the difficulty to estimate the resource costs that security researchers bear when searching for bugs (e.g., time spent). This information would help further test and understand our results, which show that there are physical limitations regarding the possibility for an individual to find an arbitrarily large number of bugs. Information on the cost functions would also bring further insights on refined expected utility functions by security researchers, in particular the distinction between expected monetary rewards and effort devoted to reputation seeking.\\

We may also further question how bug bounty program operations impact the motivation of researchers: for example, bug bounty programs may be temporarily overloaded with submissions \cite{zhao2014exploratory,zhao2015empirical}. This overload may stem from priority queueing \cite{maillart2011quantification} and effort required to verify and remediate security incidents internally \cite{kuypers2016empirical}. Delays and contingencies, such as timing and discounting effects, contribute to increase transaction costs and uncertainties for security researchers. Deeper understanding of the dynamics associated with bug bounty program operations may help establish a benchmark on the performance of organization designs and their implementations.\\

Finally, there is evidence that security researchers have specialized knowledge and skills. The competitive environment associated with bug bounty programs reinforces incentives to specialize. Two types of specialization exist in bug bounty programs: {\it program-specific} and {\it vulnerability-specific} \cite{zhao2014exploratory,zhao2015empirical}. Program-specific specialization is associated with knowledge, experience and skills required to find vulnerabilities in websites and software products in one particular program. Since specialization is relatively unique to the program, a specialized researcher has fewer options to switch between programs. Vulnerability-specific specialization is associated with knowledge and skills of a particular type of vulnerability, which can exist in many different products. These researchers have stronger incentive to explore different bug bounty programs. Specialization must be accounted for when implementing organization design recommendations. In some circumstances, it may be desirable to attract a crowd of diverse yet specialized security researchers. Depending on the specialization required, targeting specific security researchers may however restrict the diversity of the resource pool. Specialization is directly associated with the concept of {\it skills and mindset}, which we have introduced here to explain the observed hard limits regarding the number of bugs a single researcher can find. This notion deserves a more thorough definition as well as a testable theory.\\





