Bug bounty programs offer a modern platform for organizations to crowdsource their software security, and for security researchers to be fairly rewarded for the vulnerabilities they find. However, little is known on the incentives set by bug bounty programs -- how they drive new bug discoveries, and how they may improve security through the progressive exhaustion of discoverable vulnerabilities. This article investigates the strategic interactions among the managers and participants of bug bounty programs. At the level of a single bug bounty program, security researchers face the St-Petersburg paradox: the probability of finding additional bugs decays rapidly, and is difficult to compensate for with increasing monetary rewards. Furthermore, bug bounty program managers have an incentive to gather the largest possible crowd of experts, which increases competition among security researchers. As a result, we find that researchers have strong incentives to switch to newly launched programs, for which a reserve of {\it low-hanging fruit} vulnerabilities is still available. Our results shed light on the technical and economic mechanisms underlying the dynamics of bug bounty program contributions, and help inform the mechanism design of bug bounty programs as they get increasingly adopted by cybersecurity-savvy organizations.
%Bug bounty programs offer a modern platform for organizations to crowdsource their software security and for security researchers to be fairly rewarded for the vulnerabilities they find. Little is known however on the incentives set by bug bounty programs: How they drive new bug discoveries, and how they supposedly improve security through the progressive exhaustion of discoverable vulnerabilities. Here, we recognize that bug bounty programs create tensions, for organizations running them on the one hand, and for security researchers on the other hand. At the level of one bug bounty program, security researchers face a sort of St-Petersburg paradox: The probability of finding additional bugs decays fast, and thus can hardly be matched with a sufficient increase of monetary rewards. Furthermore, bug bounty program managers have an incentive to gather the largest possible crowd to ensure a larger pool of expertise, which in turn increases competition among security researchers. As a result, we find that researchers have high incentives to switch to newly launched programs, for which a reserve of {\it low-hanging fruit} vulnerabilities is still available. Our results inform on the technical and economic mechanisms underlying the dynamics of bug bounty program contributions, and may in turn help improve the mechanism design of bug bounty programs that get increasingly adopted by cybersecurity savvy organizations.