Bug bounty programs offer a modern way for organizations to crowdsource their software security, and for security researchers to be fairly rewarded for the vulnerabilities they find. However, little is known on the incentives set by bug bounty programs -- how they drive engagement and new bug discoveries. This article provides an empirical investigation of the strategic interactions among the managers and participants of bug bounty programs, as well as the intermediation by bug bounty platforms. We find that for a given bug bounty program, each security researcher can only expect to discover a bounded number of bugs. This result offers a validation step to a theory brought forth early on by Brady et al. \cite{brady1999murphy}, which proposes that each security researcher inspecting a software piece offers a unique environment of skills and mindset, which is prone to the discovery of bugs that others may not be able to uncover. Bug bounty programs indeed benefit from the engagement of large crowds of researchers. Conversely, security researchers benefit greatly from searching bugs in multiple bug bounty programs. We find however that, following a strong front-loading effect, newly launched programs attract researchers, at the expense of older programs: the probability of finding bugs decays as $\sim 1/t^{0.4}$ after the launch of a program, even though bugs found later yield on average higher reward. Our results lead to us to formulate 3 recommendations for organizing bug bounty programs and platforms: (i) organize enrollment, mobility and renewal of security researchers across bounty programs, (ii) feature and organize programs for front-loading, and (iii) organize fluid market transactions to reduce uncertainty and thus, reduce incentives for security researchers to sell on the black market.



