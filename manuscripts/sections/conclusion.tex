\section{Conclusion}
\label{sec:conclusion}
In this paper, we have investigated how crowds of security researchers hunt software bugs and vulnerabilities and report to a bug bounty platform. Consistent with the famous adage ``Given enough eyeballs, all bugs are shallow" by Eric Raymond \cite{raymond1999cathedral}, we have found that security researchers face challenging difficulties when trying to uncover large numbers of bugs in a same bounty program: The super-linear reward increase for each new bug found does not counterbalance the sharply decreasing probability of finding new bugs by the same person. This result is consistent with the theory proposed by Brady et al. on maximized entropy of bug discovery as an evolutionary process, following adaptation to changing environments \cite{brady1999murphy}: Each security researcher tests software within an environment bounded by her skills and mindset. This result brings a fundamental justification for the existence of markets for bugs, beyond internalized security operations and research: Bug bounty programs offer a way to capitalize on these {\it environments} provided by the involvement of many security researchers. Yet difficulties for researchers to find large numbers of bugs in one bug bounty program bring incentives for mobility across programs. In particular, we find that the launch of new bug bounty programs has a negative effect on incumbent programs regarding bug submissions. We thus bring forth 3 organization design recommendations. First, mobility shall be encouraged across bug bounty programs and perhaps across bug bounty platforms. Second, one major duty of bug bounty platforms shall be to enroll large number of security researchers, and guide them towards low-hanging fruits, which for new researchers, may not necessarily be new bug bounty programs. Finally, bug bounty program manager shall feature significant changes in the codebase, such as major releases, to help security researchers focus on issues they have not been exposed previously. Such strategy shall help retain more long-term engagement, while not crowding out motivation in face of difficulty. 
