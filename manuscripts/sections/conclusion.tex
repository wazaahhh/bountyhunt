\section{Conclusion}
\label{sec:conclusion}
In this paper, we have investigated how crowds of security researchers hunt software vulnerabilities and how they report their findings to bug bounty programs on dedicated online platforms. Consistent with the famous adage ``Given enough eyeballs, all bugs are shallow'' by Eric Raymond \cite{raymond1999cathedral}, we have found that security researchers face challenging difficulties when trying to uncover large numbers of bugs in the same bounty program: the super-linear reward increase for newly discovered bugs does not counterbalance the sharply decreasing probability of finding new bugs by the same person. This result is consistent with the theory proposed by Brady et al. on maximized entropy of bug discovery as an evolutionary process, following adaptation to changing environments \cite{brady1999murphy}: each security researcher tests software within an environment bounded by her skills and mindset. This result brings a fundamental justification for the existence of markets for bugs, beyond internalized security operations and research: bug bounty programs offer a way to capitalize on these diverse {\it environments} provided by the involvement of many security researchers. Yet, difficulties for researchers to find large numbers of bugs in one bug bounty program bring incentives for mobility across programs. In particular, we find that the launch of new bug bounty programs has a negative effect on incumbent programs regarding bug submissions. We thus propose three organization design recommendations. First, enrollment, mobility and renewal shall be encouraged across bug bounty programs as well as across bug bounty platforms. Second, bug bounty program managers shall highlight major changes in the codebase and get organized for front-loading, in order to help security researchers focus on recent codebase changes. Finally, recognizing that market structures are a powerful mechanism to mobilize large crowds of security researchers, we recommend to organize fluid market transactions, in order to reduce as much as possible uncertainties associated with bug submissions.



